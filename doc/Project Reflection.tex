\documentclass[a4paper]{article}

\title{Project Reflection}
\author{Joshua Kearns}
\date{\today}

\begin{document}

\maketitle

\clearpage
\tableofcontents
\clearpage

\section{Project Quality}
	\subsection{NUClear}
		NUClear was the core part of the project, without this architectural framework our project would have been of no use to the NUBots team. NUClear was a huge success. NUClear has acheived its goals of being a fast, easy to use, flexible, loosely coupled framework. It has used innovative techniques such as template metaprogramming and compile time message routing to acheive this. It to allow the Nubots system to be much more expandible and flexible once NUClearPort is complete. It has also allowed for the easy construction of completely new systems such as Robot Dance and Mech Warrior using components already available from NUClearPort. It is also already being used for research purposes by the Nubots team.
	\subsection{NUClearPort}
		While NUClearPort did not progress as far as orginally intended it is well on it's way to converting the old Nubots soccer system to NUClear. It has given the Nubots team a great starting point from which they intend to complete the tranfer to the NUClear which they now be using. Rather than directly converting the old components to NUClear, our team was able to upgrade many components. NUClear has enabled new vital sytems to be upgraded including networking, and debugging which will greatly improve further development. NUClear has enabled the Nubots system to make use of all processor cores on future robot hardware. The addition of roles has made it incredibly easy to various different system with little effort. Roles can be used for easily creating new unit tests for components or even to allow Nubots system components to be used for completely different purposes such as Robot Dance. While NUClearPort is not complete enough for the robot to play soccer, it is well on its way to doing so for the 2014 RoboCup, as well as adding many new useful features.
	\subsection{Robot Dance}
		Robot Dance has acheived its purpose of being able to dance in time to the beats of music. The core of the system is solid, though some components could use improvement. While the beat tracker works great for music with simple beats, it struggles with complicated music with more difficult beats. Additionally the audio input from microphone does not currently take into account noise from the robots fan and motors which is a major problem. Fortunately with the modular design of NUClear and the new roles system it will be easy to replace the current beat tracker with a better version. It will also be possible to add a filter component inbetween the audio input and beat tracking to take care of the problem of noise. Due to time constraints we were unable to implement these upgrades.
\section{Engineering Approaches}
\section{Overall Development}

Content

\end{document}
